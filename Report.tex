\documentclass[]{article}
\usepackage{lmodern}
\usepackage{amssymb,amsmath}
\usepackage{ifxetex,ifluatex}
\usepackage{fixltx2e} % provides \textsubscript
\ifnum 0\ifxetex 1\fi\ifluatex 1\fi=0 % if pdftex
  \usepackage[T1]{fontenc}
  \usepackage[utf8]{inputenc}
\else % if luatex or xelatex
  \ifxetex
    \usepackage{mathspec}
  \else
    \usepackage{fontspec}
  \fi
  \defaultfontfeatures{Ligatures=TeX,Scale=MatchLowercase}
\fi
% use upquote if available, for straight quotes in verbatim environments
\IfFileExists{upquote.sty}{\usepackage{upquote}}{}
% use microtype if available
\IfFileExists{microtype.sty}{%
\usepackage{microtype}
\UseMicrotypeSet[protrusion]{basicmath} % disable protrusion for tt fonts
}{}
\usepackage[margin=0.25in]{geometry}
\usepackage{hyperref}
\hypersetup{unicode=true,
            pdftitle={edx - Movie Lens Project},
            pdfauthor={Hesham Al Kayed},
            pdfborder={0 0 0},
            breaklinks=true}
\urlstyle{same}  % don't use monospace font for urls
\usepackage{color}
\usepackage{fancyvrb}
\newcommand{\VerbBar}{|}
\newcommand{\VERB}{\Verb[commandchars=\\\{\}]}
\DefineVerbatimEnvironment{Highlighting}{Verbatim}{commandchars=\\\{\}}
% Add ',fontsize=\small' for more characters per line
\usepackage{framed}
\definecolor{shadecolor}{RGB}{248,248,248}
\newenvironment{Shaded}{\begin{snugshade}}{\end{snugshade}}
\newcommand{\KeywordTok}[1]{\textcolor[rgb]{0.13,0.29,0.53}{\textbf{#1}}}
\newcommand{\DataTypeTok}[1]{\textcolor[rgb]{0.13,0.29,0.53}{#1}}
\newcommand{\DecValTok}[1]{\textcolor[rgb]{0.00,0.00,0.81}{#1}}
\newcommand{\BaseNTok}[1]{\textcolor[rgb]{0.00,0.00,0.81}{#1}}
\newcommand{\FloatTok}[1]{\textcolor[rgb]{0.00,0.00,0.81}{#1}}
\newcommand{\ConstantTok}[1]{\textcolor[rgb]{0.00,0.00,0.00}{#1}}
\newcommand{\CharTok}[1]{\textcolor[rgb]{0.31,0.60,0.02}{#1}}
\newcommand{\SpecialCharTok}[1]{\textcolor[rgb]{0.00,0.00,0.00}{#1}}
\newcommand{\StringTok}[1]{\textcolor[rgb]{0.31,0.60,0.02}{#1}}
\newcommand{\VerbatimStringTok}[1]{\textcolor[rgb]{0.31,0.60,0.02}{#1}}
\newcommand{\SpecialStringTok}[1]{\textcolor[rgb]{0.31,0.60,0.02}{#1}}
\newcommand{\ImportTok}[1]{#1}
\newcommand{\CommentTok}[1]{\textcolor[rgb]{0.56,0.35,0.01}{\textit{#1}}}
\newcommand{\DocumentationTok}[1]{\textcolor[rgb]{0.56,0.35,0.01}{\textbf{\textit{#1}}}}
\newcommand{\AnnotationTok}[1]{\textcolor[rgb]{0.56,0.35,0.01}{\textbf{\textit{#1}}}}
\newcommand{\CommentVarTok}[1]{\textcolor[rgb]{0.56,0.35,0.01}{\textbf{\textit{#1}}}}
\newcommand{\OtherTok}[1]{\textcolor[rgb]{0.56,0.35,0.01}{#1}}
\newcommand{\FunctionTok}[1]{\textcolor[rgb]{0.00,0.00,0.00}{#1}}
\newcommand{\VariableTok}[1]{\textcolor[rgb]{0.00,0.00,0.00}{#1}}
\newcommand{\ControlFlowTok}[1]{\textcolor[rgb]{0.13,0.29,0.53}{\textbf{#1}}}
\newcommand{\OperatorTok}[1]{\textcolor[rgb]{0.81,0.36,0.00}{\textbf{#1}}}
\newcommand{\BuiltInTok}[1]{#1}
\newcommand{\ExtensionTok}[1]{#1}
\newcommand{\PreprocessorTok}[1]{\textcolor[rgb]{0.56,0.35,0.01}{\textit{#1}}}
\newcommand{\AttributeTok}[1]{\textcolor[rgb]{0.77,0.63,0.00}{#1}}
\newcommand{\RegionMarkerTok}[1]{#1}
\newcommand{\InformationTok}[1]{\textcolor[rgb]{0.56,0.35,0.01}{\textbf{\textit{#1}}}}
\newcommand{\WarningTok}[1]{\textcolor[rgb]{0.56,0.35,0.01}{\textbf{\textit{#1}}}}
\newcommand{\AlertTok}[1]{\textcolor[rgb]{0.94,0.16,0.16}{#1}}
\newcommand{\ErrorTok}[1]{\textcolor[rgb]{0.64,0.00,0.00}{\textbf{#1}}}
\newcommand{\NormalTok}[1]{#1}
\usepackage{longtable,booktabs}
\usepackage{graphicx,grffile}
\makeatletter
\def\maxwidth{\ifdim\Gin@nat@width>\linewidth\linewidth\else\Gin@nat@width\fi}
\def\maxheight{\ifdim\Gin@nat@height>\textheight\textheight\else\Gin@nat@height\fi}
\makeatother
% Scale images if necessary, so that they will not overflow the page
% margins by default, and it is still possible to overwrite the defaults
% using explicit options in \includegraphics[width, height, ...]{}
\setkeys{Gin}{width=\maxwidth,height=\maxheight,keepaspectratio}
\IfFileExists{parskip.sty}{%
\usepackage{parskip}
}{% else
\setlength{\parindent}{0pt}
\setlength{\parskip}{6pt plus 2pt minus 1pt}
}
\setlength{\emergencystretch}{3em}  % prevent overfull lines
\providecommand{\tightlist}{%
  \setlength{\itemsep}{0pt}\setlength{\parskip}{0pt}}
\setcounter{secnumdepth}{5}
% Redefines (sub)paragraphs to behave more like sections
\ifx\paragraph\undefined\else
\let\oldparagraph\paragraph
\renewcommand{\paragraph}[1]{\oldparagraph{#1}\mbox{}}
\fi
\ifx\subparagraph\undefined\else
\let\oldsubparagraph\subparagraph
\renewcommand{\subparagraph}[1]{\oldsubparagraph{#1}\mbox{}}
\fi

%%% Use protect on footnotes to avoid problems with footnotes in titles
\let\rmarkdownfootnote\footnote%
\def\footnote{\protect\rmarkdownfootnote}

%%% Change title format to be more compact
\usepackage{titling}

% Create subtitle command for use in maketitle
\providecommand{\subtitle}[1]{
  \posttitle{
    \begin{center}\large#1\end{center}
    }
}

\setlength{\droptitle}{-2em}

  \title{edx - Movie Lens Project}
    \pretitle{\vspace{\droptitle}\centering\huge}
  \posttitle{\par}
    \author{Hesham Al Kayed}
    \preauthor{\centering\large\emph}
  \postauthor{\par}
      \predate{\centering\large\emph}
  \postdate{\par}
    \date{12 May 2019}

\usepackage{float}
\floatplacement{figure}{H}

\begin{document}
\maketitle

\section{Introduction}\label{introduction}

~~~ This Project is part of \emph{edx - Data Science course}\footnote{\url{https://www.edx.org/professional-certificate/harvardx-data-science}}
, in which we want to build a recommendation model using the
\emph{MovieLens} dataset. \emph{A recommender system is a subclass of
information filtering system that seeks to predict the ``rating'' or
``preference'' a user would give to an item. They are primarily used in
commercial applications.} \footnote{\url{https://en.wikipedia.org/wiki/Recommender_system}}

~~~ In our case, we want to predict the rating a user would give to a
given movie based on the provided dataset, in this project we are
provided with a pre-wrangled version of \emph{MovieLens} dataset which
can be obtained using code in \texttt{Section.4}. to build our model
this dataset will still need extra formatting as described in
\texttt{Section\ 2.1}. Our first step to build our model is to
understand data at hand; we need to explore our dataset, and find out
the distribution of our predictors and how they affect movie ratings,
\texttt{Section\ 2.2} explore the relation between between movie ratings
and other predictors.

~~~ In \texttt{Section\ 3} we build a model based on the effect, or the
average error each predictor contributed to the distance between the
actual rating and the over all average. only the \emph{user},
\emph{movie} and \emph{year} effect are considered, because of the
limitations on the used PC, and as we will see ,only using these three
parameters will achieve the required model performance.

\section{Exploratory data analysis}\label{exploratory-data-analysis}

\subsection{Data Structure}\label{data-structure}

~~~ Examining training set, we find that we have 9000055 rows and 6
columns, \texttt{Table.1} shows the header of our dataset,it is logical
to assume that each user have one review only for each movie, below code
confirms our assumption about the uniqueness of \emph{usesrId + movieId}
by showing no duplicates.

\begin{Shaded}
\begin{Highlighting}[]
\NormalTok{edx }\OperatorTok\StringTok{ }\KeywordTok{group_by}\NormalTok{(userId, movieId) }\OperatorTok\StringTok{ }\KeywordTok{summarise}\NormalTok{(}\DataTypeTok{N=}\KeywordTok{n}\NormalTok{()) }\OperatorTok\StringTok{ }\KeywordTok{filter}\NormalTok{(N }\OperatorTok{>}\DecValTok{1}\NormalTok{)}
\end{Highlighting}
\end{Shaded}

\begin{longtable}[]{@{}lrrrrll@{}}
\caption{Training Set Header}\tabularnewline
\toprule
& userId & movieId & rating & timestamp & title & genres\tabularnewline
\midrule
\endfirsthead
\toprule
& userId & movieId & rating & timestamp & title & genres\tabularnewline
\midrule
\endhead
1 & 1 & 122 & 5 & 838985046 & Boomerang (1992) &
Comedy\textbar{}Romance\tabularnewline
2 & 1 & 185 & 5 & 838983525 & Net, The (1995) &
Action\textbar{}Crime\textbar{}Thriller\tabularnewline
4 & 1 & 292 & 5 & 838983421 & Outbreak (1995) &
Action\textbar{}Drama\textbar{}Sci-Fi\textbar{}Thriller\tabularnewline
5 & 1 & 316 & 5 & 838983392 & Stargate (1994) &
Action\textbar{}Adventure\textbar{}Sci-Fi\tabularnewline
6 & 1 & 329 & 5 & 838983392 & Star Trek: Generations (1994) &
Action\textbar{}Adventure\textbar{}Drama\textbar{}Sci-Fi\tabularnewline
7 & 1 & 355 & 5 & 838984474 & Flintstones, The (1994) &
Children\textbar{}Comedy\textbar{}Fantasy\tabularnewline
\bottomrule
\end{longtable}

\emph{title} column holds the title of the movie and the year it was
aired, separation of title and year will be more useful. \emph{genres}\\
is stored as a string which can be separated by ``\textbar{}'', after
separating genres we find 20 unique genres. \emph{timestamp} should be
changed to a readable format, and for the sake of simplicity only the
year will be considered.

\begin{longtable}[]{@{}lllllll@{}}
\caption{Types and Missing Data}\tabularnewline
\toprule
& userId & movieId & rating & timestamp & title & genres\tabularnewline
\midrule
\endfirsthead
\toprule
& userId & movieId & rating & timestamp & title & genres\tabularnewline
\midrule
\endhead
Type & integer & double & double & integer & character &
character\tabularnewline
NA & 0 & 0 & 0 & 0 & 0 & 0\tabularnewline
Empty & 0 & 0 & 0 & 0 & 0 & 0\tabularnewline
\bottomrule
\end{longtable}

\texttt{Table.2} shows that training data is complete with no missing or
empty records.

\newpage

~~~

\begin{longtable}[]{@{}rrrlllr@{}}
\caption{EDX Header After Formatting}\tabularnewline
\toprule
userId & movieId & rating & timestamp & title & genres &
year\tabularnewline
\midrule
\endfirsthead
\toprule
userId & movieId & rating & timestamp & title & genres &
year\tabularnewline
\midrule
\endhead
1 & 122 & 5 & 9710 & Boomerang & c(``Comedy'', ``Romance'') &
1992\tabularnewline
1 & 185 & 5 & 9710 & Net, The & c(``Action'', ``Crime'', ``Thriller'') &
1995\tabularnewline
1 & 292 & 5 & 9710 & Outbreak & c(``Action'', ``Drama'', ``Sci-Fi'',
``Thriller'') & 1995\tabularnewline
1 & 316 & 5 & 9710 & Stargate & c(``Action'', ``Adventure'', ``Sci-Fi'')
& 1994\tabularnewline
1 & 329 & 5 & 9710 & Star Trek: Generations & c(``Action'',
``Adventure'', ``Drama'', ``Sci-Fi'') & 1994\tabularnewline
1 & 355 & 5 & 9710 & Flintstones, The & c(``Children'', ``Comedy'',
``Fantasy'') & 1994\tabularnewline
\bottomrule
\end{longtable}

\subsection{Data Description}\label{data-description}

The purposes of this study is to build a model to predict movie ratings
that will be given by a given user, so we will start there.

\begin{longtable}[]{@{}rrrrrr@{}}
\caption{Ratings Summary}\tabularnewline
\toprule
Min. & 1st Qu. & Median & Mean & 3rd Qu. & Max.\tabularnewline
\midrule
\endfirsthead
\toprule
Min. & 1st Qu. & Median & Mean & 3rd Qu. & Max.\tabularnewline
\midrule
\endhead
0.5 & 3 & 4 & 3.512465 & 4 & 5\tabularnewline
\bottomrule
\end{longtable}

\begin{longtable}[]{@{}lrrrrrrrrrr@{}}
\toprule
& 0.5 & 1 & 1.5 & 2 & 2.5 & 3 & 3.5 & 4 & 4.5 & 5\tabularnewline
\midrule
\endhead
\% & 0.95 & 3.84 & 1.18 & 7.9 & 3.7 & 23.57 & 8.8 & 28.76 & 5.85 &
15.45\tabularnewline
\bottomrule
\end{longtable}

\begin{itemize}
\tightlist
\item
  \textbf{Users tend to rate in whole numbers instead of fractions.}
\item
  \textbf{Most ratings are above 2.5 .}
\item
  \textbf{Highest rating used is 4.0 followed by 3.0 and 5.0.}
\end{itemize}

\begin{figure}

{\centering \includegraphics{Report_files/figure-latex/Numer of reviews per year-1} 

}

\caption{Numer of reviews per year}\label{fig:Numer of reviews per year}
\end{figure}

\newpage

Looking at \texttt{Fig.1}, we can see that even after breaking
\emph{rating} description in \texttt{Table.4} into a year based summary,
that the same pattern holds, which is most reviews averaged at
\textbf{4.0} followed by \textbf{3.0} and \textbf{5.0}, which shows that
in general people tends to give above average ratings. \texttt{Fig.2}
verifies this assumption.

\begin{figure}

{\centering \includegraphics{Report_files/figure-latex/unnamed-chunk-8-1} 

}

\caption{User Avrage Rating}\label{fig:unnamed-chunk-8}
\end{figure}

Also we can see that some users gave an average rating above
\textbf{4.5}, which means they like every movie they watch, and others
tend to dislike every movie they like. looking at \texttt{Fig.3}, we can
see that users who gave an average above \textbf{4.5} are more
consistent with their ratings (\emph{low standard deviation}) compared
to low average users.

\begin{figure}

{\centering \includegraphics{Report_files/figure-latex/unnamed-chunk-9-1} 

}

\caption{Rating Spread Per User}\label{fig:unnamed-chunk-9}
\end{figure}

\newpage

by examining reviews \emph{timestamps}, we can see that all reviews was
taken from 1995, 2009, looking at \texttt{Fig.4} we can see that most
users tends to review the most recent movies. and not all users are
interested in classical movies.

\begin{figure}

{\centering \includegraphics{Report_files/figure-latex/unnamed-chunk-10-1} 

}

\caption{Decade Popularity}\label{fig:unnamed-chunk-10}
\end{figure}

Examining available movies, we find that there are 10677 unique titles.
from which 6titles have perfect score. but as we can see in
\texttt{Table.6}, these movies have one or two reviews only, and if we
filter for movies with a standard deviation less than \textbf{0.01} we
get a range of Reviews per movie between 2, 4; this give us the
indication that movies with low number of reviews will give unreliable
results; filtering for movies with more than ten reviews, we get
\texttt{table.7} which makes much more sense.

\begin{longtable}[]{@{}rlrrr@{}}
\caption{Top Movies With The Highest Average}\tabularnewline
\toprule
movieId & title & Number\_Reviews & Avgerage\_Rating &
Standard\_Deviation\tabularnewline
\midrule
\endfirsthead
\toprule
movieId & title & Number\_Reviews & Avgerage\_Rating &
Standard\_Deviation\tabularnewline
\midrule
\endhead
3226 & Hellhounds on My Trail & 1 & 5 & NA\tabularnewline
33264 & Satan's Tango (Sátántangó) & 2 & 5 & 0\tabularnewline
42783 & Shadows of Forgotten Ancestors & 1 & 5 & NA\tabularnewline
51209 & Fighting Elegy (Kenka erejii) & 1 & 5 & NA\tabularnewline
53355 & Sun Alley (Sonnenallee) & 1 & 5 & NA\tabularnewline
64275 & Blue Light, The (Das Blaue Licht) & 1 & 5 & NA\tabularnewline
\bottomrule
\end{longtable}

\begin{longtable}[]{@{}rlrrr@{}}
\caption{Top Movies With The Highest Average; N \textgreater{}
10}\tabularnewline
\toprule
movieId & title & Number\_Reviews & Avgerage\_Rating &
Standard\_Deviation\tabularnewline
\midrule
\endfirsthead
\toprule
movieId & title & Number\_Reviews & Avgerage\_Rating &
Standard\_Deviation\tabularnewline
\midrule
\endhead
318 & Shawshank Redemption, The & 28015 & 4.46 & 0.72\tabularnewline
858 & Godfather, The & 17747 & 4.42 & 0.81\tabularnewline
50 & Usual Suspects, The & 21648 & 4.37 & 0.76\tabularnewline
527 & Schindler's List & 23193 & 4.36 & 0.81\tabularnewline
904 & Rear Window & 7935 & 4.32 & 0.73\tabularnewline
912 & Casablanca & 11232 & 4.32 & 0.82\tabularnewline
\bottomrule
\end{longtable}

\newpage

\begin{figure}

{\centering \includegraphics{Report_files/figure-latex/unnamed-chunk-12-1} 

}

\caption{Rating Spread Per Movie}\label{fig:unnamed-chunk-12}
\end{figure}

As we can see in \texttt{Fig.5}, movie ratings has an average standard
deviation of 0.97, and a mean of 3.21. movies that are above average
have a lower deviation compared to below average movies.

\subsection{Conclusions}\label{conclusions}

\begin{itemize}
\tightlist
\item
  Users tends to give ratings above the center of the used
  scale;\textbf{2.5}. at an average of \textbf{3.5}.
\item
  In general, the most used rating is \textbf{4.0}, and this trend is
  followed regardless of the year.
\item
  Users differ in their ratings, but it can be summarized that the more
  a user is consistent with their rating the higher the average rating
  they give, but the same does not apply for users with high deviation
  in their ratings.
\item
  New movies are more popular than classical ones, still classical
  movies have their base of users.
\item
  as in users, in general movies with low deviation have higher ratings.
\end{itemize}

\newpage

\section{Building Model}\label{building-model}

\subsection{Formulation}\label{formulation}

I am going to build a model as described in edx - Data Science: Machine
Learning - Recommendation Systems. we will set \(Y\) as the actual
rating, \(\mu\) as the over all average, \(\epsilon\) as the error or
distance from \(\mu\). which can be interpreted as, \emph{all movies
should have a rating of} \(\mu\), \emph{but for some effect}
\(\epsilon\), \emph{the rating deviates to} \(Y\).
\[Y = \mu + \epsilon \] from our data exploration we can break
\(\epsilon\) into \emph{user effect} \(b_u\), \emph{movie effect}
\(b_i\) and \emph{year effect} \(b_y\), which leaves us with. \[
Y= \mu + b_i + b_u + b_y 
\]

above model assumes that there are only three effects, which we know
from experience that is not correct, to counter that we add \(\epsilon\)
as random error. \[
Y = \mu + b_i + b_u + b_y + \epsilon  \Rightarrow Y = Y_{hat}+\epsilon
\]

\subsection{Calculating Effects}\label{calculating-effects}

by including \(b_u\) and \(b_y\) in the random error parameter, we can
assume that, \[
b_i + \epsilon  = Y - \mu 
\] so we can calculate \(b_i\) using below code,

\begin{Shaded}
\begin{Highlighting}[]
\CommentTok{# Calculationg over all rating avrage.}
\NormalTok{mu <-}\KeywordTok{mean}\NormalTok{(edx}\OperatorTok{$}\NormalTok{rating)}
\end{Highlighting}
\end{Shaded}

\begin{Shaded}
\begin{Highlighting}[]
\CommentTok{# Calculating movie effect}
\NormalTok{movie_avg <-}\StringTok{ }\NormalTok{edx }\OperatorTok\StringTok{ }
\StringTok{  }\KeywordTok{group_by}\NormalTok{(movieId) }\OperatorTok\StringTok{ }
\StringTok{  }\KeywordTok{summarise}\NormalTok{(}\DataTypeTok{b_i =} \KeywordTok{mean}\NormalTok{(rating }\OperatorTok{-}\StringTok{ }\NormalTok{mu))}
\end{Highlighting}
\end{Shaded}

as we now have an estimation for \(b_i\), we can estimate \(b_u\) and
keep the year effect as part of our random error parameter, \[
b_u + \epsilon = Y - \mu - b_i
\]

\begin{Shaded}
\begin{Highlighting}[]
\CommentTok{# Calculating user effect}
\NormalTok{user_avg  <-}\StringTok{ }\NormalTok{edx }\OperatorTok\StringTok{ }
\StringTok{  }\KeywordTok{left_join}\NormalTok{(movie_avg, }\DataTypeTok{by =} \StringTok{"movieId"}\NormalTok{) }\OperatorTok
\StringTok{  }\KeywordTok{group_by}\NormalTok{(userId) }\OperatorTok\StringTok{ }
\StringTok{  }\KeywordTok{summarise}\NormalTok{(}\DataTypeTok{b_u=} \KeywordTok{mean}\NormalTok{(rating }\OperatorTok{-}\StringTok{ }\NormalTok{mu }\OperatorTok{-}\StringTok{ }\NormalTok{b_i))}
\end{Highlighting}
\end{Shaded}

and the same goes for year effect \(b_y\), \[
b_y + \epsilon = Y - \mu - b_i - b_u
\]

\begin{Shaded}
\begin{Highlighting}[]
\CommentTok{# Calculating year effect}
\NormalTok{year_avg  <-}\StringTok{ }\NormalTok{edx }\OperatorTok\StringTok{ }
\StringTok{  }\KeywordTok{left_join}\NormalTok{(movie_avg, }\DataTypeTok{by =} \StringTok{"movieId"}\NormalTok{) }\OperatorTok
\StringTok{  }\KeywordTok{left_join}\NormalTok{(user_avg, }\DataTypeTok{by=} \StringTok{"userId"}\NormalTok{) }\OperatorTok
\StringTok{  }\KeywordTok{group_by}\NormalTok{(year) }\OperatorTok\StringTok{ }
\StringTok{  }\KeywordTok{summarise}\NormalTok{(}\DataTypeTok{b_y=} \KeywordTok{mean}\NormalTok{(rating }\OperatorTok{-}\StringTok{ }\NormalTok{mu }\OperatorTok{-}\StringTok{ }\NormalTok{b_i }\OperatorTok{-}\StringTok{ }\NormalTok{b_u))}
\end{Highlighting}
\end{Shaded}

Based on our assumption, we know that \(Y_{hat} = \mu +b_i+b_u+b_y\); so
we need to collect the above calculated effects and add them together,
below function will facilitates that.

\begin{Shaded}
\begin{Highlighting}[]
\CommentTok{# takes df with the same structure as edx}
\CommentTok{# returns input df binded with b_i, b_u, b_y and predicted rating y_hat}

\NormalTok{PredictRating <-}\StringTok{ }\ControlFlowTok{function}\NormalTok{(df)\{}
\NormalTok{  df }\OperatorTok
\StringTok{    }\KeywordTok{left_join}\NormalTok{(movie_avg, }\DataTypeTok{by =} \StringTok{"movieId"}\NormalTok{) }\OperatorTok\StringTok{ }\CommentTok{# collect movie effect}
\StringTok{    }\KeywordTok{left_join}\NormalTok{(user_avg, }\DataTypeTok{by=}\StringTok{"userId"}\NormalTok{) }\OperatorTok\StringTok{     }\CommentTok{# collect user effect}
\StringTok{    }\KeywordTok{left_join}\NormalTok{(year_avg, }\DataTypeTok{by =} \StringTok{"year"}\NormalTok{) }\OperatorTok\StringTok{     }\CommentTok{# collect year effect}
\StringTok{    }\KeywordTok{mutate}\NormalTok{(}\DataTypeTok{y_hat =}\NormalTok{ mu }\OperatorTok{+}\StringTok{ }\NormalTok{b_i  }\OperatorTok{+}\StringTok{ }\NormalTok{b_u }\OperatorTok{+}\StringTok{ }\NormalTok{b_y,    }\CommentTok{# calculate y_hat}
           \DataTypeTok{y_hat =} \KeywordTok{if_else}\NormalTok{(y_hat }\OperatorTok{>}\StringTok{  }\DecValTok{5}\NormalTok{ ,  }\FloatTok{5.0}\NormalTok{, y_hat), }\CommentTok{# make sure that our predictions are within range}
           \DataTypeTok{y_hat =} \KeywordTok{if_else}\NormalTok{(y_hat }\OperatorTok{<}\StringTok{ }\FloatTok{0.5}\NormalTok{ , }\FloatTok{0.5}\NormalTok{, y_hat))\} }\CommentTok{# make sure that our predictions are within range}
\end{Highlighting}
\end{Shaded}

\subsection{Model Performance}\label{model-performance}

Model performance is measured in \texttt{RMSE} \footnote{\url{https://en.wikipedia.org/wiki/Root-mean-square_deviation}},
where \(N\) is the number of records.

\[
RMSE = \sqrt {\frac{\sum_{}^{N} (Y - Y_{hat})^2}{N}}
\]

\begin{Shaded}
\begin{Highlighting}[]
\NormalTok{edx }\OperatorTok\StringTok{ }\KeywordTok{PredictRating}\NormalTok{()}
\end{Highlighting}
\end{Shaded}

\begin{longtable}[]{@{}lrr@{}}
\caption{Model Performance; Test Set}\tabularnewline
\toprule
Included.Effects & RMSE & Improvement\tabularnewline
\midrule
\endfirsthead
\toprule
Included.Effects & RMSE & Improvement\tabularnewline
\midrule
\endhead
mu & 1.0603313 & 0.00\tabularnewline
mu + b\_i & 0.9423475 & 11.13\tabularnewline
mu + b\_i + b\_u & 0.8567039 & 9.09\tabularnewline
mu + b\_i + b\_u + b\_y & 0.8563777 & 0.04\tabularnewline
\bottomrule
\end{longtable}

RMSE score is under the required \textbf{0.9}.

apply the same formatting to the test set, then applying \emph{predict
function} we get.

\begin{Shaded}
\begin{Highlighting}[]
\NormalTok{validation}\OperatorTok\StringTok{ }\KeywordTok{PredictRating}\NormalTok{()}
\end{Highlighting}
\end{Shaded}

\begin{longtable}[]{@{}lrr@{}}
\caption{Model Performance; Test Set}\tabularnewline
\toprule
Included.Effects & RMSE & Improvement\tabularnewline
\midrule
\endfirsthead
\toprule
Included.Effects & RMSE & Improvement\tabularnewline
\midrule
\endhead
mu & 1.0612018 & 0.00\tabularnewline
mu + b\_i & 0.9439087 & 11.05\tabularnewline
mu + b\_i + b\_u & 0.8653488 & 8.32\tabularnewline
mu + b\_i + b\_u + b\_y & 0.8650043 & 0.04\tabularnewline
\bottomrule
\end{longtable}

\begin{figure}

{\centering \includegraphics{Report_files/figure-latex/Error-1} 

}

\caption{Error Distribution, training set}\label{fig:Error}
\end{figure}

\newpage

\section{Reference Code - Get
Datasets}\label{reference-code---get-datasets}

\begin{Shaded}
\begin{Highlighting}[]
\NormalTok{#############################################################}
\CommentTok{# Create edx set, validation set, and submission file}
\NormalTok{#############################################################}

\CommentTok{# Note: this process could take a couple of minutes}

\ControlFlowTok{if}\NormalTok{(}\OperatorTok{!}\KeywordTok{require}\NormalTok{(tidyverse)) }\KeywordTok{install.packages}\NormalTok{(}\StringTok{"tidyverse"}\NormalTok{, }\DataTypeTok{repos =} \StringTok{"http://cran.us.r-project.org"}\NormalTok{)}
\ControlFlowTok{if}\NormalTok{(}\OperatorTok{!}\KeywordTok{require}\NormalTok{(caret)) }\KeywordTok{install.packages}\NormalTok{(}\StringTok{"caret"}\NormalTok{, }\DataTypeTok{repos =} \StringTok{"http://cran.us.r-project.org"}\NormalTok{)}

\CommentTok{# MovieLens 10M dataset:}
\CommentTok{# https://grouplens.org/datasets/movielens/10m/}
\CommentTok{# http://files.grouplens.org/datasets/movielens/ml-10m.zip}

\NormalTok{dl <-}\StringTok{ }\KeywordTok{tempfile}\NormalTok{()}
\KeywordTok{download.file}\NormalTok{(}\StringTok{"http://files.grouplens.org/datasets/movielens/ml-10m.zip"}\NormalTok{, dl)}

\NormalTok{ratings <-}\StringTok{ }\KeywordTok{read.table}\NormalTok{(}\DataTypeTok{text =} \KeywordTok{gsub}\NormalTok{(}\StringTok{"::"}\NormalTok{, }\StringTok{"}\CharTok{\textbackslash{}t}\StringTok{"}\NormalTok{, }\KeywordTok{readLines}\NormalTok{(}\KeywordTok{unzip}\NormalTok{(dl, }\StringTok{"ml-10M100K/ratings.dat"}\NormalTok{))),}
                      \DataTypeTok{col.names =} \KeywordTok{c}\NormalTok{(}\StringTok{"userId"}\NormalTok{, }\StringTok{"movieId"}\NormalTok{, }\StringTok{"rating"}\NormalTok{, }\StringTok{"timestamp"}\NormalTok{))}

\NormalTok{movies <-}\StringTok{ }\KeywordTok{str_split_fixed}\NormalTok{(}\KeywordTok{readLines}\NormalTok{(}\KeywordTok{unzip}\NormalTok{(dl, }\StringTok{"ml-10M100K/movies.dat"}\NormalTok{)), }\StringTok{"}\CharTok{\textbackslash{}\textbackslash{}}\StringTok{::"}\NormalTok{, }\DecValTok{3}\NormalTok{)}
\KeywordTok{colnames}\NormalTok{(movies) <-}\StringTok{ }\KeywordTok{c}\NormalTok{(}\StringTok{"movieId"}\NormalTok{, }\StringTok{"title"}\NormalTok{, }\StringTok{"genres"}\NormalTok{)}
\NormalTok{movies <-}\StringTok{ }\KeywordTok{as.data.frame}\NormalTok{(movies) }\OperatorTok\StringTok{ }\KeywordTok{mutate}\NormalTok{(}\DataTypeTok{movieId =} \KeywordTok{as.numeric}\NormalTok{(}\KeywordTok{levels}\NormalTok{(movieId))[movieId],}
                                           \DataTypeTok{title =} \KeywordTok{as.character}\NormalTok{(title),}
                                           \DataTypeTok{genres =} \KeywordTok{as.character}\NormalTok{(genres))}

\NormalTok{movielens <-}\StringTok{ }\KeywordTok{left_join}\NormalTok{(ratings, movies, }\DataTypeTok{by =} \StringTok{"movieId"}\NormalTok{)}

\CommentTok{# Validation set will be 10% of MovieLens data}

\KeywordTok{set.seed}\NormalTok{(}\DecValTok{1}\NormalTok{)}
\NormalTok{test_index <-}\StringTok{ }\KeywordTok{createDataPartition}\NormalTok{(}\DataTypeTok{y =}\NormalTok{ movielens}\OperatorTok{$}\NormalTok{rating, }\DataTypeTok{times =} \DecValTok{1}\NormalTok{, }\DataTypeTok{p =} \FloatTok{0.1}\NormalTok{, }\DataTypeTok{list =} \OtherTok{FALSE}\NormalTok{)}
\NormalTok{edx <-}\StringTok{ }\NormalTok{movielens[}\OperatorTok{-}\NormalTok{test_index,]}
\NormalTok{temp <-}\StringTok{ }\NormalTok{movielens[test_index,]}

\CommentTok{# Make sure userId and movieId in validation set are also in edx set}

\NormalTok{validation <-}\StringTok{ }\NormalTok{temp }\OperatorTok\StringTok{ }
\StringTok{  }\KeywordTok{semi_join}\NormalTok{(edx, }\DataTypeTok{by =} \StringTok{"movieId"}\NormalTok{) }\OperatorTok
\StringTok{  }\KeywordTok{semi_join}\NormalTok{(edx, }\DataTypeTok{by =} \StringTok{"userId"}\NormalTok{)}

\CommentTok{# Add rows removed from validation set back into edx set}

\NormalTok{removed <-}\StringTok{ }\KeywordTok{anti_join}\NormalTok{(temp, validation)}
\NormalTok{edx <-}\StringTok{ }\KeywordTok{rbind}\NormalTok{(edx, removed)}

\KeywordTok{rm}\NormalTok{(dl, ratings, movies, test_index, temp, movielens, removed)}
\end{Highlighting}
\end{Shaded}


\end{document}
